%Original version in April 2002 by Antje Endemann
%
%Adapted for usage in Student Conference course by jubo (050118).
%Optimized for usage with pdflatex. For usage with plain latex,
%figures must be in Postscript (.ps or .eps files) and additional
%packages must be imported.
%Last updated 100105 by jubo
%
%This template works for outline and annotated bibliography
%(deliverable 2) without any changes.
%For usage with full and final papers (deliverables 3 and 4a/b)
%you must make several changes. All necessary changes are
%described in three specific commented sections preceded by
%`%%%%% BEGIN OF ADJUSTMENTS SECTION %%%%%'.
%The changes are in order of appearance:
%   - adjust the `\documentclass' command
%   - uncomment the abstract section
%   - adjust the `\bibliographystyle' command

%%%%% BEGIN OF 1st ADJUSTMENTS SECTION %%%%%
% You must use only one of the following `\documentclass' commands
% For the outline and annotated bibliography (deliverable 2)
% you must use the following command:
%
\documentclass[runningheads,a4paper,oribibl]{llncs}
%
%For the full and final paper (deliverables 3 and 4a/b)
% you must use the command below (and put the one above in comments).
% In other words, the 'oribib' option must be used for Deliverable 2
% but not for 3 and 4a/b.
% 
%
%\documentclass[runningheads,a4paper]{llncs}
%
%%%%% END OF 1st ADJUSTMENTS SECTION %%%%%

\usepackage[T1]{fontenc}  %% needed for special characters (umlaut)
\usepackage[latin1]{inputenc}
\usepackage{graphicx}     %% for graphical things such as including pictures
\usepackage{url}          %% for proper formatting of URLs
\usepackage{parskip}

\begin{document}
\pagestyle{plain}

\mainmatter
% https://sv.sharelatex.com/learn/Text_alignment

\title{Mobile and contactless payments impact on consumers spending}
\author{Moa Hermansson}
\institute{
    Department of Applied Physics and Electronics \\
    Ume\aa   University, Sweden \\
    \email{mohe0025@umu.se}
}
\maketitle

%\newpage

% The abbreviated title will be shown in the headers of even pages.
% You should use the full title unless it is too long.
%\titlerunning{Example Outline and Annotated Bibliography}
%\maketitle

%%%%% BEGIN OF 2nd ADJUSTMENTS SECTION %%%%%
% Do NOT include an abstract in the outline and annotated bibliography
% (deliverable 2). For the full and final paper (deliverables 3 and 4a/b),
% you must uncomment the following two commands and place your text for
% the abstract in between them.
%
% \begin{abstract}
%   Here goes the actual text of your abstract.
% \end{abstract}
%
%%%%% END OF 2nd ADJUSTMENTS SECTION %%%%%

 \begin{abstract}
 Abstract text
\end{abstract}


\section{Introduction}

The amount of payment alternatives are increasing and there are a number of different options to choose from. 85\% of Sweden's population (above age 11) owns a smartphone, Apple and Android are competing with each other and has released payment options for smartphones. Cash payments are being replaced with card, contactless cards, bank transfers and mobile payments. Sweden is on its way to become a cashless society, in 2015 cash transactions made up 2\% of the value of all payments made in Sweden~\cite{busarticleApplePay}.
Swish payment application was released in Sweden 2012 and reached 6 million users in 2017.

The hit of the mobile payments technology in the U.S. began in 2014 when Apple Pay was released, because at the time it was the only major mobile wallet available on market.
The following year Android pay was released. Both products allow shoppers to make instant payments using their phones like a contactless card.

The amount of contactless transactions with Mastercard in Europe have increased with 116\% in 2017 compared to previous year.  By beginning of 2018 half of Sweden's payment terminals will accept contactless payments [1]. Making it possible for faster and simpler transactions than ever before. So what are the consequences for the consumers. Do the possibility of fast transactions impact on consumers spending.


\subsection{Objective}
The objective of this paper is to investigate whether Swedish consumers are more willing to spend money when paying with mobile- and contactless payments than with card or cash. 

\section{Theory}

There are three kinds of payment methods that are the most common options for retail consumers to use when paying and these are cash, card and mobile payment. Two of the payment methods will be in focus, card- and mobile payments.

\subsection{The Technology}

RFID is a method of uniquely identifying items using radio waves. RFID consists of a tag, a reader and an antenna. Through the antenna a interrogating signal is sent and the tag responds with its unique information. There are active or passive RFID tags. Passive tags do not have an internal power source but is instead powered by the electromagnetic energy transmitted from an RFID reader. Active RFID tags use battery-powered RFID and continuously broadcast their own signal. Active tags provide a much longer read range than passive ones. But the active tags are also a lot more expensive \cite{nfc}.

Passive RFID tags can operate at three different frequencies of radio waves. Low Frequency(LF) 125 - 134 KHz, High Frequency(HF) 13,56 MHz and Ultra High Frequency(UHF) 865-960 MHz.

The High frequency(HF) has a typical read range of about 1 centimeter up to 1 meter and is also called Near-Field Communication (NFC). This frequency is used with data transmission, access control applications and passport security.

So NFC operates at the same frequency as HF RFID readers and tags. A NFC device can act both as a reader and a tag which has made NFC a popular option for contactless payments. NFC devices can read passive NFC tags and some NFC devices can even read passive HF RFID tags. NFC is used in credit cards, debit card and key fobs and other things. Most smartphones have NFC technology built in and NFC is being used with both Apple pay and Samsung pay \cite{smartphone}.

\subsection{Mobile payments}

98\% of the Swedish population has a mobile phone and 85\% of the population has a smartphone in 2017. It has increased with 8\% compared to 2015 when 77\% of the Swedish population had a smartphone.~\cite{soi2017}


One of the most popular mobile payments methods in Sweden are an application named Swish payments. The application has grown in popularity over the recent years, in 2015 44\% of the Swedish population which owns a smart phone uses Swish[1]. In 2017 it is 80\% of the Swedes who owns a smart phone that uses Swish.


\subsection{Contactless payments}
Contactless payment systems are credit- and debit cards.
Contactless bank cards started around 2005 in the U.S. and was released in some parts of Europe in 2006. 
Contactless payment cards was not released in Sweden until year 2014.

Contactless cards are similar to traditional payment cards but In contactless cards there is an embedded integrated circuits. The circuits can communicate with a terminal through NFC, Near-field communication. Near-field communication enables two electronic devices to communicate by having them within 4 cm of each other[4]. 

Paying with a contactless card means that you hold the card against the payment terminal instead of putting the chip in. In Sweden you can make purchases up to 200 kr without having to put the pincode in the terminal. Making it easier than ever to make small transactions and it has even created some concern with the safety of the cards.

\subsection{Cash and card payments}


\section{Method}

A survey was created to evaluate the use of mobile- and contactless payments, which of them are mostly used and where.
The survey was a web based form and was distributed over social media. A pilot test (will) be conducted to evaluate and develop the form itself. In the first section of the survey the users answers some basic questions about their age, gender and occupation.

Questions:

Which of the following payment methods do you use?
Preferred methods and usage.

Do you own a contactless card and if, usage and what kind of purchases do you use it for?

What payment method do you prefer when paying online?
If that payment method had a fee, would you still use it?

Do you have the application swish and if, usage and what purpose?

How much of your payments is by credit or debit card?

Would you still use mobile- and contactless card if it had a fee?

Do you think the wider range of payment options has had an impact on consumers spending?

Do you think the wider range of payment options has had an impact on your spending?




\section{Result}

\subsection{Survey result}
\section{Discussion}
\subsection{Future work}
\section{Conclusions}

%%%%% BEGIN OF 3rd ADJUSTMENTS SECTION %%%%%
% For the outline and annotated bibliography (deliverable 2)
% you must use the following two commands:
%
\nocite{*}  % Includes ALL entries from the .bib-file, even if they are
            % not '\cite{}'-ed in the text above
\bibliographystyle{plain-annote}
%
%For the full and final paper (deliverables 3 and 4a/b)
% use ONLY the following commands (i.e. put the commands above into
% comments and uncomment the command below):
%
% \bibliographystyle{splncs}
%
%%%%% END OF ADJUSTMENTS SECTION %%%%%

\bibliography{demo}

\end{document}
